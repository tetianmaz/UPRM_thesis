\chapter{Tracker DQM}

Maintaining the integrity of the tracking performance of the collected data is foundational. The current monitoring process is handled manually by people. With the HL-LHC upgrade, the number of MEs and event rates will increase significantly, which will make the manual process even harder. The NMF approach can be used to automate some of these processes for anomaly detection.

\section{NMF-Based Model for Anomaly Detection}

NMF is a dimensionality reduction technique that factorizes input matrix \( X \in \mathbb{R}_{\geq 0}^{m \times n} \) into two lower-dimensional non-negative matrices: a coefficient matrix \( W \in \mathbb{R}_{\geq 0}^{m \times r} \) and a basis matrix \( H \in \mathbb{R}_{\geq 0}^{r \times n} \), such that:

\begin{equation}
    X_{m,n} \approx \sum_{r=1}^{k} W_{m,r} H_{r,n}
\end{equation}

Here, \( r \) is the number of components or latent features. The input matrix \( X \) represents histograms for a particular ME per LS, for example, charge distribution or cluster size. The rows of the matrix \( H \) represent the basis histograms, while \( W \) contains the weights corresponding to each feature used for reconstructing the data. During the training process, the Frobenius norm is used as an error for minimizing the difference between the input and reconstructed data:

\begin{equation}
    \min_{W, H \geq 0} \| X - WH \|_F^2
\end{equation}

After the basis matrix \( H \) and coefficient matrix \( W \) have been trained, test data \( X_{\text{test}} \) can be found in the same basis:

\begin{equation}
    W_{\text{test}} = \arg \min_{W \geq 0} \| X_{\text{test}} - WH \|_F^2
\end{equation}

The first component extracts dominant trends of the input data, and the remaining components extract variations, shifts in the distributions, noise, or handle the reconstruction of the tail of the Landau-like distribution. If the secondary components (all except the first one) have large contributing weights, or if the reconstruction error is significant, then the LS is anomalous.

Such anomalies may stem from a variety of sources, including:
\begin{itemize}
    \item Failing pixels or ROC-related issues,
    \item Trigger rate fluctuations,
    \item Synchronization problems within the Tracker readout system,
    \item Loss of hits or incomplete event data,
    \item and other detector or DAQ-related irregularities.
\end{itemize}
